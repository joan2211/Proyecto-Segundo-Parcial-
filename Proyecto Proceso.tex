\documentclass[12pt,a4paper]{article}
\usepackage[utf8]{inputenc}
\usepackage[spanish]{babel}
\usepackage{amsmath}
\usepackage{amsfonts}
\usepackage{amssymb}
\usepackage{makeidx}
\usepackage{graphicx}
\usepackage{lmodern}
\usepackage{kpfonts}
\usepackage{fourier}
\usepackage[left=2cm,right=2cm,top=2cm,bottom=2cm]{geometry}
\author{Joan Moran - Carolina González}
\title{Proyecto}
\begin{document}
\maketitle Proyecto Segundo Parcial
proceso
\newpage
1.	Normas ISO\\
ISO (Organización Internacional de Normalización) es una federación mundial de organismos nacionales de normalización (organismos miembros de ISO). El trabajo de preparación de las normas internacionales normalmente se realiza a través de los comités técnicos de ISO. Cada organismo miembro interesado en una materia para la cual se haya establecido un comité técnico, tiene el derecho de estar representado en dicho comité. Las organizaciones internacionales, públicas y privadas, en coordinación con ISO, también participan en el trabajo. ISO colabora estrechamente con la Comisión Electrotécnica Internacional (IEC) en todas las materias de normalización electrotécnica.\\
1.2	Norma ISO 9001 \\
La Norma ISO 9001 especifica los requisitos para un sistema de gestión de la calidad que pueden utilizarse para su aplicación interna por las organizaciones, para certificación o con fines contractuales. Se centra en la eficacia del sistema de gestión de la calidad para satisfacer los requisitos del cliente.\\
ara certificación o con fines contractuales. Se centra en la eficacia del sistema de gestión de la calidad 

\end{document}